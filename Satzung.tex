\documentclass{article}
\usepackage[T1]{fontenc}                
\usepackage[utf8]{inputenc} 
\usepackage[ngerman]{babel} 
\usepackage{enumerate}
\usepackage{geometry}
\usepackage{titlesec}
\usepackage{hyperref}

\title{Satzung der Fachschaft Molekulare Biomedizin}
\date{}

\providecommand{\tightlist}{\setlength{\itemsep}{0pt}\setlength{\parskip}{0pt}}
\geometry {a4paper, top= 20mm, bottom=20mm, left=20mm, right=20mm}
\hypersetup{hidelinks}

\renewcommand\thepart{\Alph{part}}

\titleformat{\part}{\fontsize{15pt}{15pt}\bfseries}{\thepart .\ }{0pt}{}
\titleformat{\chapter}{\fontsize{15pt}{15pt}\bfseries}{\thechapter .\ }{0pt}{}
\titleformat{\section}{\fontsize{11pt}{13pt}\bfseries}{\S \ \thesection \ }{0pt}{\normalsize}

\begin{document}
\thispagestyle{empty} \tableofcontents \thispagestyle{empty}
\pagebreak
\clearpage
\pagenumbering{arabic}
\maketitle

\part*{Präambel}\label{pruxe4ambel}
Diese Satzung, beschlossen durch die Fachschaftsvollversammlung der Fachschaft Molekulare Biomedizin vom 14. Mai 2018, zuletzt geändert durch die Erste Änderungssatzung der Satzung der Fachschaft Molekulare Biomedizin, beschlossen durch die Fachschaftsvollversammlung vom 05.04.2019, regelt die Rechte und Pflichten der selbstverwalteten Organe der Fachschaft Molekulare Biomedizin der Rheinischen Friedrich-Wilhelms-Universität Bonn.

\part{Fachschaft}\label{a.-fachschaft}

\section{Begriffsbestimmung und Rechtsstellung}\label{begriffsbestimmung-und-rechtsstellung}

\begin{enumerate}[(1)]
	\item Alle Studierenden, die in den folgenden Studienfächern im Hauptfach an der Rheinischen Friedrich- Wilhelms Universität (RFWU) Bonn eingeschrieben sind, bilden die Fachschaft Molekulare Biomedizin:
	\begin{itemize}
		\tightlist
		\item Biochemistry (Master of Science)
		\item Immunobiology: from molecules to integrative systems (Master of Science)
		\item Life and Medical Sciences (Master of Science)
		\item Medical Immunosciences and Infection (Master of Science)
		\item Molekulare Biomedizin (Bachelor of Science)
		\item Molekulare Biomedizin (Promotion)
	\end{itemize}
	\item Die Fachschaft nimmt alle sie betreffenden Aufgaben innerhalb der Studierendenschaft wahr und vertritt im Rahmen ihrer Möglichkeiten die Belange der Studierenden, die in den oben genannten Studienfächern eingeschrieben sind.
\end{enumerate}

\section{Organe der Fachschaft}\label{organe-der-fachschaft}

\begin{enumerate}[(1)]
	\item Die Fachschaft äußert ihren Willen durch ihre Organe und deren Wahl.
	\item Organe der Fachschaft sind:
	\begin{enumerate}[1.]
		\tightlist
		\item die Fachschaftsvertretung (FSV)
		\item der Fachschaftsrat (FSR),
		\item die Fachschaftsvollversammlung (FSVV),
		\item die Fachausschüsse (FA)
		\item die Studienfachvollversammlung (SfVV)
	\end{enumerate}
	\item Die Amtszeit der unter § 2 Abs. 2 Nr. 1,2 und 4 aufgeführten Organe beträgt ein Jahr. Bis zur Neuwahl der Nachfolgemitglieder bleiben die Mitglieder der betreffenden Organe kommissarisch im Amt.
\end{enumerate}

\section{Gemeinsame Aufgaben der Organe FSV, FSR und FA}\label{gemeinsame-aufgaben-der-organe-fsv-fsr-und-fa}

\begin{enumerate}[(1)]
	\item Die Fachschaft fördert auf der Grundlage der verfassungsmäßigen Ordnung die politische Bildung und das staatsbürgerliche Verantwortungsbewusstsein der Mitglieder der Fachschaft.
	\item Die Organe FSV, FSR und FA vertreten die hochschulpolitischen Belange der Fachschaft und beziehen Stellung zu hochschulpolitischen Fragen. Eine über die Aufgaben der Organe FSV, FSR und FA hinausgehende allgemeinpolitische Willensbildung vollzieht sich in den studentischen Vereinigungen der Hochschule.
	\item FSV, FSR und FA wirken an der fachlichen und organisatorischen Gestaltung des Studiums mit.
\end{enumerate}

\part{Die Fachschaftsvertretung (FSV)}\label{i.-die-fachschaftsvertretung-fsv}

\section{Rechtsstellung der FSV}\label{rechtsstellung-der-fsv}
Die FSV ist das Beschlussorgan der Studierendenschaft am Fachbereich. (§ 77 S. 2 HG in Verbindung mit § 27 Abs. 3 der Satzung der Studierendenschaft) Die FSV trifft, sollte kein Beschluss der FSVV vorliegen, alle Entscheidungen von grundlegender oder gehobener Bedeutung für die Fachschaft, die über den regulären Geschäftsbetrieb des FSR hinausgehen.

\section{Zusammensetzung und Zusammentritt der FSV}\label{zusammensetzung-und-zusammentritt-der-fsv}

\begin{enumerate}[(1)]
	\item Die FSV besteht aus elf gewählten Mitgliedern.
	\item Sie tritt mindestens dreimal im Semester zusammen. Für die Einladung zu einer FSV-Sitzung gilt die Schriftform. Die Einladung durch unsignierte elektronische Form (E-Mail) ist gegen den ausgesprochenen Willen eines Mitglieds der FSV nicht zulässig, in diesem Fall hat das FSV- Präsidium das Mitglied in signierter Schriftform einzuladen.
	\item Die Mitglieder der FSV sind grundsätzlich verpflichtet, an den Sitzungen teilzunehmen, sofern sie nicht begründet entschuldigt sind. Fehlt ein FSV-Mitglied unentschuldigt, oder wurde eine Entschuldigung weniger als 24h vor der Sitzung schriftlich vorgebracht, so muss das FSV-Mitglied zur nächsten Sitzung einen Kuchen oder eine äquivalente Süßspeise mitbringen.
	\item Wurde nach einem unentschuldigten Fehlen keine angemessene Süßspeise mitgebracht, wird dem betreffenden FSV-Mitglied für eine Sitzung das Stimmrecht entzogen. Über die Angemessenheit muss die FSV auf Antrag abstimmen.
\end{enumerate}

\section{Wahl der FSV}\label{wahl-der-fsv}
\begin{enumerate}[(1)]
	\item Die FSV wird jährlich von den Mitgliedern der Fachschaft in allgemeiner, unmittelbarer, freier, gleicher und geheimer Urnenwahl gewählt.
	\item Die Wahl wird vom Wahlausschuss der Fachschaft vorbereitet und durchgeführt.
	\item Der Wahlausschuss ist spätestens bis zum 30. Tag vor dem ersten Wahltag durch die FSV zu wählen. Die Wahl des Wahlausschusses ist in der Sitzungseinladung anzukündigen.
	\item Der Wahlausschuss besteht aus dem Wahlleiter und mindestens zwei weiteren Mitgliedern. Für den Fall des Rücktritts eines Mitgliedes des Wahlausschusses muss dieser schriftlich beim FSV-Präsidium beantragt werden und das Mitglied bleibt bis zur Wahl eines Nachfolgers kommissarisch im Amt. Mitglieder des Wahlausschusses, auch jene, die kommissarisch im Amt sind, dürfen für die Wahl nicht kandidieren.
	\item In der Sitzungseinladung für die Wahl des Wahlausschusses ist explizit auf § 26 Abs. 2 der Fachschaftswahlordnung (FSWO) hinzuweisen, welcher das Recht auf Beantragung einer personalisierten Verhältniswahl regelt.
	\item Der Wahlleiter beruft die konstituierende Sitzung der neu gewählten FSV ein und leitet sie, bis ein Vorsitzender gewählt ist.
	\item Das Nähere bestimmt die Fachschaftswahlordnung.
\end{enumerate}

\section{Aufgaben und Zuständigkeit der FSV}\label{aufgaben-und-zustuxe4ndigkeit-der-fsv}

\begin{enumerate}[(1)]
	\item Die FSV wählt den FSR.
	\item Die FSV wählt den Kassenprüfungsausschuss.
	\item Die FSV wählt den Wahlausschuss.
	\item Die FSV beschließt über den Haushaltsplan.
	\item Die FSV beschließt mit der Mehrheit ihrer satzungsmäßigen Mitglieder die politische und finanzielle Entlastung des FSR. Die finanzielle Entlastung kann nicht verweigert werden, wenn die Kassenprüfung keine Ungenauigkeiten ergibt. Die Entlastung muss von einem Mitglied der FSV beantragt werden. Finanzielle Entlastung kann auch von den Kassenprüfern beantragt werden. Auf Antrag eines Mitglieds der FSV muss eine Einzelentlastung durchgeführt werden.
	\item Für die FSV gilt die Geschäftsordnung des Studierendenparlaments entsprechend, soweit anwendbar, sofern sie sich keine eigene Geschäftsordnung gibt.
\end{enumerate}

\section{Das Präsidium der FSV und seine Aufgaben}\label{das-pruxe4sidium-der-fsv-und-seine-aufgaben}

\begin{enumerate}[(1)]
	\item Das Präsidium der FSV besteht aus
	\begin{enumerate}[1.]
		\tightlist
		\item dem Vorsitzenden,
		\item dem stellvertretenden Vorsitzenden,
		\item dem Schriftführer.
	\end{enumerate}
	\item Alle Mitglieder des Präsidiums müssen FSV-Mitglieder sein und werden einzeln in öffentlicher Wahl in der konstituierenden Sitzung gewählt. Auf Antrag eines einzelnen Mitglieds der FSV muss die Wahl in geheimer Form stattfinden (vgl.: § 6 Abs.6).
	\item Die Ämter des Präsidiums der FSV sind unvereinbar mit der Mitgliedschaft im FSR.
	\item Der kommissarische Status des FSR-Vorsitzenden lässt eine auf einer FSV-Sitzung erfolgende Wahl ins Präsidium der FSV zu, wenn in derselben Sitzung ein Nachfolger für das Amt des FSR- Vorsitzenden gewählt wird.
	\item Zur Wahl des Präsidiums bedarf es der einfachen Mehrheit der satzungsgemäßen Mitglieder der FSV. Erhält im ersten Wahlgang kein Kandidat die notwendige Stimmenzahl, so findet unverzüglich ein zweiter Wahlgang statt. Erreicht auch in diesem Wahlgang kein Kandidat die notwendige Stimmenzahl, so gilt im dritten Wahlgang der Kandidat als gewählt, der die relative Mehrheit der Stimmen auf sich vereint. Während einer Wahl mit mehreren Wahlgängen können neue Kandidaten nur für die Wahlliste vorgeschlagen werden, wenn die Mehrheit der anwesenden Mitglieder einem Antrag auf Öffnung der Wahlliste zustimmt.
	\item Mitglieder des Präsidiums können nur mit der absoluten Mehrheit der Stimmen der FSV-Mitglieder durch die Wahl eines Nachfolgers abberufen werden.
	\item Der FSV-Vorsitzende führt die laufenden Geschäfte der FSV. Er ist insbesondere dafür verantwortlich, die satzungsgemäße Arbeit aller Organe der Fachschaft sicherzustellen.
	\item Der Schriftführer ist für die Erstellung der Sitzungsprotokolle verantwortlich. Er kann an seiner Statt ein anwesendes Mitglied der Fachschaft für die Aufgabe des Protokollanten nominieren. Der Schriftführer ist dafür verantwortlich, dass das Protokoll der FSV-Sitzung spätestens eine Woche nach der Sitzung sowohl in digitaler Form ausgefertigt an den FSV-Vorsitzenden weitergeleitet wird und spätestens nach zwei Wochen öffentlich einsehbar ist. Dem Protokoll ist eine Anwesenheitsliste der jeweiligen FSV-Sitzung hinzuzufügen.
	\item Über die Vollständigkeit und Richtigkeit des Protokolls wird in der nachfolgenden FSV-Sitzung mit der Mehrheit der anwesenden Mitglieder abgestimmt. Zuvor hat jedes Mitglied der Fachschaft das Recht, eine Stellungnahme zum Protokoll abzugeben.
	\item Tritt ein Mitglied des Präsidiums zurück, wählt die FSV unverzüglich den Nachfolger. Kann die Wahl nicht auf derselben Sitzung erfolgen, so führt das ausgeschiedene Mitglied sein Amt kommissarisch bis zur Nachwahl weiter.
\end{enumerate}

\section{Einberufung der FSV}\label{einberufung-der-fsv}

\begin{enumerate}[(1)]
	\item Der Vorsitzende der FSV führt ihre laufenden Geschäfte. Er beruft die FSV ein, wenn
	\begin{enumerate}[1.]
		\tightlist
		\item der FSR-Sprecher,
		\item die Mehrheit des FSR,
		\item sechs Mitglieder der FSV,
		\item die FSVV,
		\item ein FA,
		\item eine SfVV,
		\item 5\% der Mitglieder der Fachschaft dies unter Angabe von zu behandelnden Tagesordnungspunkten schriftlich verlangen.
	\end{enumerate}
	\item Die Einladung muss sieben Tage vor der geplanten Sitzung an alle FSR-, FA- und FSV-Mitglieder verschickt werden. Maßgeblich ist der Eingangszeitpunkt bzw. das Datum des Poststempels. Zu demselben Termin muss auch öffentlich eingeladen werden.
\end{enumerate}

\section{Ausscheiden, Ausschluss und Nachrücken von Mitgliedern}\label{ausscheiden-ausschluss-und-nachruxfccken-von-mitgliedern}

\begin{enumerate}[(1)]
	\item Ein Mitglied scheidet aus der FSV aus
	\begin{enumerate}[1.]
		\tightlist
		\item durch Niederlegung seines Mandats,
		\item durch Exmatrikulation oder durch Umschreibung in ein anderes Hauptfach,
		\item durch rechtskräftige Disziplinarstrafe,
		\item durch Tod.
	\end{enumerate}
	\item Ist ein FSV-Mitglied während einer Sitzung dreimal zur Ordnung gerufen worden und beim zweiten Mal auf die Folgen eines dritten Rufes zur Ordnung hingewiesen worden, so schließt der FSV-Vorsitzende die Person von der Sitzung aus.
	\item Bei Wiederbesetzung eines freigewordenen Sitzes können solange Personen nachrücken, bis sich die Kandidatenliste erschöpft hat.
\end{enumerate}

\section{Beschlüsse der FSV}\label{beschluxfcsse-der-fsv}

\begin{enumerate}[(1)]
	\item Rederecht haben alle Mitglieder der Fachschaft Molekulare Biomedizin.
	\item Stimm- und Antragsrecht haben nur FSV-Mitglieder.
	\item Auf schriftlichen Antrag von mindestens drei Mitgliedern der FSV hat ein betreffendes FSR-Mitglied während der den Antrag betreffenden nachfolgenden Sitzung anwesend zu sein (Zitierrecht).
	\item Ein Beschluss ist rechtmäßig zustande gekommen, wenn
	\begin{enumerate}[1.]
		\tightlist
		\item Die Sitzung der FSV fristgerecht einberufen wurde,
		\item die FSV beschlussfähig war und
		\item er die relative Mehrheit gefunden hat, soweit die Satzung nichts anderes vorschreibt.
	\end{enumerate}
	\item Die FSV gilt solange als beschlussfähig, bis auf Antrag eines FSV-Mitgliedes durch den Vorsitzenden das Gegenteil festgestellt wird.
	\item Die Beschlussfähigkeit wird auf Antrag unverzüglich festgestellt. Sie ist gegeben, wenn mehr als die Hälfte der FSV-Mitglieder anwesend ist. Ein Einspruch gegen diesen Antrag ist nicht möglich. Der FSV- Vorsitzende überprüft die Beschlussfähigkeit durch namentlichen Aufruf.
	\item Bei Beschlussunfähigkeit muss nach spätestens 14 Tagen eine zweite Sitzung mit der gleichen Tagesordnung einberufen werden. Die normalen Ladungsfristen sind zu wahren. Die Einladung hat ausdrücklich darauf hinzuweisen, dass diese Sitzung unabhängig von der Zahl der anwesenden Mitglieder beschlussfähig ist.
	\item FSV-Beschlüsse der laufenden Sitzungsperiode können durch Beschluss mit einer absoluten 2/3 Mehrheit aufgehoben werden.
\end{enumerate}

\section{Ausschüsse der FSV}\label{ausschuxfcsse-der-fsv}

\begin{enumerate}[(1)]
	\item Die FSV wählt die Mitglieder des Wahlausschusses, sowie den Vorsitzenden als Wahlleiter und die Stellvertreter mit der Mehrheit der satzungsmäßigen Mitglieder. Es ist die Aufgabe des Wahlausschusses, die Voraussetzungen für einen möglichst reibungslosen Ablauf der Wahl sowie eine hohe Wahlbeteiligung zu schaffen. Die Wahl kann auch per Briefwahl erfolgen. Näheres regelt die Fachschaftswahlordnung.
	\item Die FSV wählt als Mitglieder des Kassenprüfungsausschusses drei Kassenprüfer mit absoluter Mehrheit. Die Kassenprüfer müssen Mitglieder der Fachschaft sein. Mitglieder des FSR, der FSV oder eines FA im zu prüfenden Haushaltsjahr können nicht zum Kassenprüfer gewählt werden. Die Kassenprüfer kontrollieren die ordnungsgemäße Kassenführung des Haushaltsjahres für dessen Kontrolle sie gewählt wurden und erstatten der FSV über das Ergebnis der Prüfung Bericht.
	\item Ist ein oder sind mehrere FA vorgesehen und gewählt, so ist umgehend ein Aufgabenverteilungs- und Haushaltsauschuss zu konstituieren. Dieser Ausschuss setzt sich zusammen aus dem Vorsitzenden und dem Finanzreferenten des FSR sowie dem oder den Vorsitzenden des oder der FA. Der Finanzreferent des FSR hat den Vorsitz, leitet die Sitzung und konstituiert den Ausschuss. Der Ausschuss beschließt über den Haushaltsplanentwurf und die Aufgabenverteilung zwischen FSR und dem oder den FA mit qualifizierter Mehrheit, sofern der Vorsitzende und der Finanzreferent des FSR mit der Mehrheit stimmen.
\end{enumerate}

\section{Vorlesungsfreie Zeit}\label{vorlesungsfreie-zeit}

\begin{enumerate}[(1)]
	\item Die Regelungen über die FSV gelten auch in der vorlesungsfreien Zeit.
\end{enumerate}

\part{Der Fachschaftsrat (FSR)}\label{ii.-der-fachschaftsrat-fsr}

\section{Rechtsstellung des FSR}\label{rechtsstellung-des-fsr}

\begin{enumerate}[(1)]
	\item Der FSR vertritt die Fachschaft und führt die Geschäfte der Fachschaft unter der Leitung seines Vorsitzenden.
\end{enumerate}

\section{Zusammensetzung des FSR}\label{zusammensetzung-des-fsr}

\begin{enumerate}[(1)]
	\item Der FSR besteht aus neun Mitgliedern.
	\item Zusätzlich können gemäß § 27 Abs. 5 Satzung der Studierendenschaft je Studienfach bis zu zwei weitere Personen durch die FSV in den FSR gewählt werden, die durch den FA dieses Fachs vorgeschlagen werden.
	\item Der FSR besteht aus
	\begin{enumerate}[1.]
		\tightlist
		\item dem Vorsitzenden,
		\item dem stellvertretenden Vorsitzenden,
		\item und dem Finanzreferenten
	\end{enumerate}
	als geschäftsführendem Vorstand, sowie sechs weiteren regulären Mitgliedern, die Studienfachvertreter nicht eingeschlossen.
	\item Der FSR tritt in öffentlicher Sitzung zusammen:
	\begin{enumerate}[1.]
		\tightlist
		\item während der Vorlesungszeit grundsätzlich 14-tägig (jeden zweiten Dienstag)
		\item auf eigenen Beschluss,
		\item auf Beschluss der FSV,
		\item auf Beschluss der FSVV,
		\item auf Beschluss eines FA.
	\end{enumerate}
	\item Auf das Zusammentreten des FSR soll in Form einer schriftlichen öffentlichen Ankündigung durch den Vorsitzenden bzw. seinen Stellvertreter hingewiesen werden.
	\item Die Mitglieder des FSR sind grundsätzlich verpflichtet, an den Sitzungen teilzunehmen, sofern sie nicht begründet entschuldigt sind. Fehlt ein FSR-Mitglied unentschuldigt, oder wurde eine Entschuldigung weniger als 24h vor der Sitzung schriftlich vorgebracht, so muss das FSR-Mitglied zur nächsten Sitzung einen Kuchen oder eine äquivalente Süßspeise mitbringen.
	\item Wurde nach einem unentschuldigten Fehlen keine angemessene Süßspeise mitgebracht, wird dem betreffenden FSR-Mitglied für eine Sitzung das Stimmrecht entzogen. Über die Angemessenheit muss der FSR auf Antrag abstimmen.
	\item Der FSR ist verpflichtet, während der Sitzungen Protokoll zu führen. Dazu ist in jeder Sitzung durch den Vorsitzenden ein Schriftführer aus den gewählten Mitgliedern zu bestimmen. Der Schriftführer ist dafür verantwortlich, dass das Protokoll der FSR-Sitzung spätestens zur nächsten FSR-Sitzung allen Mitgliedern in digitaler Form weitergeleitet wird.
	\item Dem Protokoll ist eine Anwesenheitsliste der jeweiligen FSR-Sitzung hinzuzufügen. Die Protokolle sind an geeigneter Stelle im Internet zur Verfügung zu stellen.
	\item Jedes FSR-Mitglied hat das Recht, eine Stellungnahme zum Protokoll abzugeben. Gleiches gilt für andere Fachschaftsmitglieder, die auf der Sitzung anwesend sind. Über die Vollständigkeit und Richtigkeit des Protokolls wird auf Antrag in der jeweils folgenden FSR-Sitzung mit der Mehrheit der anwesenden FSR-Mitglieder abgestimmt.
	\item Für den FSR gilt die Geschäftsordnung des Studierendenparlaments soweit anwendbar, falls er sich keine eigene Geschäftsordnung gibt.
\end{enumerate}

\section{Ausscheiden, Ausschluss und Nachrücken von Mitgliedern des FSR}\label{ausscheiden-ausschluss-und-nachruxfccken-von-mitgliedern-des-fsr}

\begin{enumerate}[(1)]
	\item Ein Mitglied scheidet aus dem FSR aus
	\begin{enumerate}[1.]
		\tightlist
		\item durch Niederlegung seines Mandats,
		\item durch Exmatrikulation oder durch Umschreibung in ein anderes Hauptfach,
		\item durch Abwahl nach §17(6),
		\item bei Unvereinbarkeit der FSR-Mitgliedschaft nach § 12 Abs. 2 Satzung der Studierendenschaft,
		\item durch Auflösung des FSR durch die FSV oder die FSVV,
		\item durch Tod,
		\item durch rechtskräftige Disziplinarstrafe.
	\end{enumerate}
	\item FSR-Mitglieder können jederzeit zurücktreten. Sie sind jedoch verpflichtet, die Geschäfte bis zur Wahl eines Nachfolgers weiterzuführen.
\end{enumerate}

\section{Wahl des FSR}\label{wahl-des-fsr}

\begin{enumerate}[(1)]
	\item Der zu wählende FSR-Sprecher muss der FSV zum Zeitpunkt seiner Wahl angehören. Der FSR- Sprecher hat das alleinige Vorschlagsrecht für alle übrigen zu wählenden Mitglieder des FSR. Mitglieder des FSR müssen eines der Studienfächer, deren Studenten durch die Fachschaft vertreten werden im Hauptfach studieren.
	\item Die Mitgliedschaft im FSR ist unvereinbar mit Ämtern des Präsidiums der FSV. Ämter im amtierenden geschäftsführenden Vorstand sind mit Ämtern des Kassenprüfungsausschusses nicht vereinbar.
	\item Der geschäftsführende Vorstand wird entsprechend § 8 Abs. 5 gewählt.
	\item Die weiteren Mitglieder des FSR neben dem geschäftsführenden Vorstand werden, auf Verlangen einzeln, mit der Mehrheit der satzungsmäßigen Mitglieder der FSV gewählt (§ 8 Abs. 5).
	\item Die FSV kann den FSR-Vorsitzenden nur im Wege eines konstruktiven Misstrauensvotums abwählen. Mit der Beendigung der Amtszeit des FSR-Vorsitzenden endet die Amtszeit aller Referenten.
	\item Nur der FSR-Vorsitzende hat das Recht, der FSV anzutragen, einen Referenten zu entlassen. Die Abwahl eines Referenten erfolgt mit absoluter 2/3 Mehrheit.
	\item Wenn es nach Entscheidung des FSR-Vorsitzenden keinen Nachfolger in diesem Amt geben soll, hat der Referent das Amt in möglichst drei Wochen ordnungsgemäß zu Ende zu führen. Tritt ein Mitglied des geschäftsführenden Vorstandes zurück, wählt die FSV unverzüglich seinen Nachfolger. Dazu muss gemäß § 9 Abs. 2 eingeladen werden.
\end{enumerate}

\section{Beschlüsse des FSR}\label{beschluxfcsse-des-fsr}

\begin{enumerate}[(1)]
	\item Rederecht haben alle Mitglieder der Fachschaft Molekulare Biomedizin.
	\item Stimm- und Antragsrecht haben nur FSR-Mitglieder.
	\item Ein Beschluss ist rechtmäßig zustande gekommen, wenn
	\begin{enumerate}[1.]
		\tightlist
		\item der FSR beschlussfähig war und
		\item er die relative Mehrheit gefunden hat, soweit die Satzung nichts anderes vorschreibt.
	\end{enumerate}
	\item Der FSR gilt solange als beschlussfähig, bis auf Antrag eines FSR-Mitgliedes durch den Vorsitzenden das Gegenteil festgestellt wird.
	\item Die Beschlussfähigkeit wird auf Antrag unverzüglich festgestellt. Sie ist gegeben, wenn mehr als die Hälfte der FSR-Mitglieder anwesend ist. Ein Einspruch gegen diesen Antrag ist nicht möglich. Der FSR- Vorsitzende überprüft die Beschlussfähigkeit durch namentlichen Aufruf.
	\item Bei Beschlussunfähigkeit muss nach spätestens 14 Tagen eine zweite Sitzung mit der gleichen Tagesordnung einberufen werden. Die normalen Ladungsfristen sind zu wahren. Die Einladung hat ausdrücklich darauf hinzuweisen, dass diese Sitzung unabhängig von der Zahl der anwesenden Mitglieder beschlussfähig ist.
	\item Ist ein FSR-Mitglied während einer Sitzung dreimal zur Ordnung gerufen worden und beim zweiten Mal auf die Folgen eines dritten Rufes zur Ordnung hingewiesen worden, so schließt der FSR-Vorsitzende die Person von der Sitzung aus.
\end{enumerate}

\section{Aufgaben und Zuständigkeiten des FSR}\label{aufgaben-und-zustuxe4ndigkeiten-des-fsr}

\begin{enumerate}[(1)]
	\item Der FSR-Vorsitzende bestimmt die Richtlinien der Arbeit des FSR und trägt dafür die Verantwortung. Innerhalb dieser Richtlinien ist jedes FSR-Mitglied gegenüber dem Vorsitzenden für sein Aufgabengebiet verantwortlich.
	\item Der FSR-Vorsitzende ist insbesondere dafür verantwortlich, die Arbeit der Organe der Fachschaft an alle Mitglieder der Fachschaft zu kommunizieren. Dazu hat er in angemessener Form regelmäßige Bekanntmachungen zu erstellen, in denen auch an die Fachschaft gerichtete Angebote und Mitteilungen kommuniziert werden. Außerdem hat die Zugänglichkeit der Sitzungsprotokolle aller Organe der Fachschaft sicherzustellen.
	\item Der FSR-Vorsitzende hat Beschlüsse, Unterlassungen oder Maßnahmen der FSV, des FSR, der FSVV, sowie eines FA, oder einer SfVV, sofern sie gegen geltendes Recht verstoßen, zu beanstanden.
	\item Der FSR kann durch Mehrheitsbeschluss Aufgabengebiete an Referenten vergeben. Mindestens sind aber Referenten für die folgenden Aufgabengebiete zu wählen:
	\begin{enumerate}[1.]
		\item ein Vertreter für die Organe der Studierendenschaft (Fachschaftenkonferenzen),
		\item Vertreter für die Gremien der Institute und Fakultät (Fachgruppe und Prüfungsausschüsse),
		\item Mindestens ein Referent für IT
		\item Mindestens ein Referent für Öffentlichkeitsarbeit,
		\item Mindestens ein Referent für die Planung von Veranstaltungen,
		\item Mindestens ein Referent für die Fachschaftsbücherei
		\item Drei Referenten für das Bier- und Kuchenkomitee
		\item Mindestens ein Referent für die Vernetzung mit Fachschaften an anderen Universitäten.
	\end{enumerate}
\end{enumerate}

\section{Vorlesungsfreie Zeit}\label{vorlesungsfreie-zeit-1}

\begin{enumerate}[(1)]
	\item Die Regelungen über den FSR gelten auch in der vorlesungsfreien Zeit.
\end{enumerate}

\part{Die Fachschaftsvollversammlung (FSVV)}\label{iii.-die-fachschaftsvollversammlung-fsvv}

\section{Rechtsstellung der FSVV}\label{rechtsstellung-der-fsvv}

\begin{enumerate}[(1)]
	\item Die FSVV, die aus allen wahlberechtigten Mitgliedern der Fachschaft Molekulare Biomedizin besteht, ist beschlussfassendes Organ der Fachschaft.
\end{enumerate}

\section{Einberufung und Durchführung der FSVV}\label{einberufung-und-durchfuxfchrung-der-fsvv}

\begin{enumerate}[(1)]
	\item Der Vorsitzende des FSR beruft die FSVV ein:
	\begin{enumerate}[1.]
		\tightlist
		\item auf Beschluss der FSV
		\item auf Beschluss des FSR
		\item auf schriftlichen Antrag von mindestens 5\% der Mitglieder der Fachschaft, sofern der Antrag eine Tagesordnung enthält.
	\end{enumerate}
	\item Die Ankündigung der FSVV erfolgt mindestens eine Woche vor ihrer Durchführung in schriftlicher Form. Die Ankündigung enthält mindestens
	\begin{enumerate}[1.]
		\tightlist
		\item die genaue Zeit und Ortsangabe der FSVV sowie
		\item ihre Tagesordnung
	\end{enumerate}
	\item Die FSVV wählt zu Beginn jeder Versammlung einen Versammlungsleiter.
	\item Für die FSVV gilt die Geschäftsordnung des Studierendenparlaments soweit anwendbar, falls sie sich keine eigene Geschäftsordnung gibt.
\end{enumerate}

\section{Aufgaben und Zuständigkeiten der FSVV}\label{aufgaben-und-zustuxe4ndigkeiten-der-fsvv}

\begin{enumerate}[(1)]
	\item Die FSVV kann die Auflösung und Neuwahl der FSV, des FSR sowie einzelner FAs mit einer 2/3 Mehrheit beschließen. Für die Neuwahl müssen die gültigen Fristen und Regelungen dieser Satzung beachtet werden.
\end{enumerate}

\section{Beschlüsse der FSVV}\label{beschluxfcsse-der-fsvv}

\begin{enumerate}[(1)]
	\item Rede-, Stimm- und Antragsrecht haben alle Mitglieder der Fachschaft.
	\item Die Entscheidungen der FSVV binden alle Organe der Fachschaft. Die FSVV ist nur beschlussfähig, wenn mindestens 10\% aller satzungsmäßigen Mitglieder der FSVV anwesend sind.
	\item Beschlüsse der FSVV können nur durch eine weitere FSVV mit der entsprechenden Mehrheit aufgehoben werden. Die Einberufung dieser folgenden FSVV erfolgt gemäß § 22.
	\item Ein Beschluss ist rechtmäßig zustande gekommen, wenn
	\begin{enumerate}[1.]
		\tightlist
		\item die FSVV beschlussfähig war und
		\item er die einfache Mehrheit gefunden hat, soweit die Satzung nichts anderes vorschreibt.
	\end{enumerate}
	\item Bei Beschlussunfähigkeit muss nach spätestens 14 Tagen eine zweite Sitzung mit der gleichen Tagesordnung einberufen werden. Die normalen Ladungsfristen sind zu wahren. Die Einladung hat ausdrücklich darauf hinzuweisen, dass diese Sitzung unabhängig von der Zahl der anwesenden Mitglieder beschlussfähig ist.
\end{enumerate}

\part{Die Studienfachvollversammlung (SfVV)}\label{iii.-die-studienfachvollversammlung-sfvv}

\section{Rechtsstellung der SfVV}\label{rechtsstellung-der-sfvv}

\begin{enumerate}[(1)]
	\item Die SfVV, die aus allen wahlberechtigten Mitgliedern des jeweiligen Studienfaches besteht, ist beschlussfassendes Organ der Mitglieder des Studienfaches.
\end{enumerate}

\section{Aufgaben der SfVV}\label{aufgaben-der-sfvv}

\begin{enumerate}[(1)]
	\item Die SfVV kann mit einfacher Mehrheit die Einrichtung eines Fachausschusses für ihr Studienfach beschließen. In diesem Fall wählt sie aus ihren Mitgliedern bis zu 5 Personen in den Fachausschuss.
\end{enumerate}

\section{Einberufung und Durchführung der SfVV}\label{einberufung-und-durchfuxfchrung-der-sfvv}

\begin{enumerate}[(1)]
	\item Der Vorsitzende des FA, ansonsten der Vorsitzende des FSR beruft die SfVV ein:
	\begin{enumerate}[1.]
		\tightlist
		\item auf Beschluss des FA,
		\item auf schriftlichen Antrag von mindestens 5\% der Mitglieder des Studienfaches, sofern der Antrag eine Tagesordnung enthält.
	\end{enumerate}
	\item Die Ankündigung der SfVV erfolgt mindestens eine Woche vor ihrer Durchführung. Die Ankündigung enthält mindestens
	\begin{enumerate}[1.]
		\tightlist
		\item die genaue Zeit und Ortsangabe der SfVV sowie
		\item ihre Tagesordnung.
	\end{enumerate}
	\item Die SfVV wählt zu Beginn jeder Versammlung einen Versammlungsleiter. Der Versammlungsleiter teilt dem FSR-Vorsitzenden die gewählten Mitglieder des FA mit, sofern eine Wahl stattfand.
	\item Für die SfVV gilt die Geschäftsordnung des Studierendenparlaments soweit anwendbar, falls sie sich keine eigene Geschäftsordnung gibt.
\end{enumerate}

\section{Beschlüsse der SfVV}\label{beschluxfcsse-der-sfvv}

\begin{enumerate}[(1)]
	\item Die SfVV ist nur beschlussfähig, wenn mindestens 10\%, aber nicht weniger als sechs, aller satzungsmäßigen Mitglieder der SfVV anwesend sind.
\end{enumerate}

\part{Der (Studien-) Fachausschuss (FA)}\label{iv.-der-studien--fachausschuss-fa}

\section{Rechtsstellung des FA}\label{rechtsstellung-des-fa}

\begin{enumerate}[(1)]
	\item Der FA vertritt die Mitglieder des jeweiligen Studienfachs innerhalb des Fachbereichs gegenüber der Professorenschaft und der Universität.
	\item Im Übrigen vertritt der FA die Mitglieder des jeweiligen Studienfachs und führt deren Geschäfte unter Leitung seines Vorsitzenden, soweit ihm durch den FSR weitergehende Vertretungs- und Geschäftsführungsbefugnisse erteilt wurden.
\end{enumerate}

\section{Zusammensetzung des FA}\label{zusammensetzung-des-fa}

\begin{enumerate}[(1)]
	\item Der FA besteht aus bis zu 5 Mitgliedern,
	\begin{enumerate}[1.]
		\tightlist
		\item dem Vorsitzenden,
		\item dem stellvertretenden Vorsitzenden
		\item und höchstens drei weiteren Mitgliedern.
	\end{enumerate}
	\item Der FA tritt in öffentlicher Sitzung zusammen:
	\begin{enumerate}
		\tightlist
		\item während der Vorlesungszeit grundsätzlich 14-tägig an einem auf der konstituierenden Sitzung des FA festgelegten Wochentag,
		\item auf eigenen Beschluss,
		\item auf Beschluss durch FSR, FSV, SfVV oder FSVV.
	\end{enumerate}
	\item Auf das Zusammentreten des FA soll in Form einer schriftlichen, öffentlichen Ankündigung durch den Vorsitzenden bzw. seinen Stellvertreter hingewiesen werden.
	\item Die Mitglieder des FA sind grundsätzlich verpflichtet, an den Sitzungen teilzunehmen, sofern sie nicht begründet entschuldigt sind.
	\item Der FA ist verpflichtet, während der Sitzungen Protokoll zu führen. Dazu ist zu Beginn jeder Sitzung durch den Vorsitzenden ein Schriftführer zu bestimmen.
	\item Für den FA gilt die Geschäftsordnung des Studierendenparlaments soweit anwendbar, falls er sich keine eigene Geschäftsordnung gibt.
\end{enumerate}

\section{Wahl des FA}\label{wahl-des-fa}

\begin{enumerate}[(1)]
	\item Auf der SfVV werden bis zu fünf Mitglieder für den FA gewählt. Die Kandidaten müssen in dem betreffenden Studienfach zum Zeitpunkt der Wahl eingeschrieben sein. Jedes anwesende Studienfachmitglied kann eine Stimme vergeben. Auf Antrag eines einzelnen Anwesenden Mitglieds der SfVV hat die Wahl in geheimer Form statt zu finden. Der FA setzt sich aus den fünf Kandidaten mit den meisten Stimmen zusammen, im Falle des Stimmengleichstandes wird durch den Versammlungsleiter öffentlich gelost. Im Falle von Unstimmigkeiten dient die FSV als schlichtendes Organ.
	\item Die von der SfVV gewählten Mitglieder für den FA werden von dem Versammlungsleiter umgehend dem FSR-Vorsitzenden mitgeteilt. Der FA ist spätestens einen Monat nach seiner Wahl auf Einladung des SfVV-Leiters zu konstituieren.
	\item Die Mitgliedschaft im FA ist unvereinbar mit Ämtern des Präsidiums der FSV und dem Amt des Finanzreferenten des geschäftsführenden Vorstandes des FSR. Ämter im amtierenden FA sind mit Ämtern des Kassenprüfungsausschusses nicht vereinbar.
	\item Der FA wählt mit einfacher Mehrheit seiner Mitglieder einen Vorsitzenden und einen stellvertretenden Vorsitzenden. Das Ergebnis der Wahl ist der FSV und an geeigneter Stelle öffentlich bekannt zu geben.
	\item Mitglieder des FA können jederzeit zurücktreten. Sie sind jedoch verpflichtet, die Geschäfte bis zur Wahl eines Nachfolgers weiterzuführen.
\end{enumerate}

\section{Aufgaben und Zuständigkeiten des FA}\label{aufgaben-und-zustuxe4ndigkeiten-des-fa}

\begin{enumerate}[(1)]
	\item Der FA-Vorsitzende bestimmt die Richtlinien der Arbeit des FA und trägt dafür die Verantwortung. Innerhalb dieser Richtlinien ist jedes Ausschussmitglied dem Vorsitzenden gegenüber für sein Aufgabengebiet verantwortlich. Der FA-Vorsitzende hat auf jeder SfVV sowie auf Einladung des FSR einen Bericht über den derzeitigen Stand der Ausschussarbeit zu geben. Der FA hat zum Ende seiner Amtszeit von maximal einem Jahr eine SfVV zur Neuwahl des FA einzuberufen.
\end{enumerate}

\part{Haushalts- und Wirtschaftsführung}\label{c.-haushalts--und-wirtschaftsfuxfchrung}

\section{Grundsätze und Kontrolle der Haushaltsführung}\label{grundsuxe4tze-und-kontrolle-der-haushaltsfuxfchrung}

\begin{enumerate}[(1)]
	\item Die Haushalts- und Wirtschaftsführung richtet sich nach den Vorgaben der Satzung der Studierendenschaft und der Geschäftsordnung der Fachschaftenkonferenz (FKGO). Dem Finanzreferenten obliegt die Finanzführung der Fachschaft. Er führt über alle Einnahmen und Ausgaben der Fachschaft ordnungsgemäß Buch.
	\item Der Finanzreferent hat vor Beginn des Haushaltsjahres einen ausgeglichenen Haushaltsplan aufzustellen und diesen der FSV auf einer Sitzung vor Beginn des Haushaltsjahres zur Abstimmung vorzulegen. Der Haushaltsplan muss mit der Einladung zur entsprechenden Sitzung allen Mitgliedern vorliegen. Das Haushaltsjahr beginnt am 1. April eines jeden Jahres.
	\item Überplanmäßige oder außerplanmäßige Ausgaben sind vor Inkrafttreten eines Nachtrags zum Haushaltsplan, der sie vorsieht, nur dann zulässig, wenn sie unabweisbar sind. Sie sind einer FSV unverzüglich anzuzeigen. Nachträge zum Haushaltsplan können nur für das laufende Haushaltsjahr eingebracht werden.
	\item Die Kassenprüfer der FSV führen eine Jahresabschlussprüfung durch. Die Kassenprüfung dient dem Zweck festzustellen, ob insbesondere
	\begin{enumerate}[1.]
		\tightlist
		\item der Kassen-Ist-Bestand mit dem Kassen-Soll-Bestand übereinstimmt,
		\item die Buchungen mit den im Haushaltsplan vorgesehenen Titeln übereinstimmen. Über die Kassenprüfung ist Protokoll zu führen, in das die Kassen- und Kontobestände aufzunehmen sind.
	\end{enumerate}
	\item Zur finanziellen Verpflichtung der Fachschaft sind die Unterschriften des FSR-Vorsitzenden und des Finanzreferenten oder die Unterschrift des zuständigen Referenten nach Zustimmung des FSR-Vorsitzenden und des Finanzreferenten erforderlich. Der FSR kann gegen die Stimmen von FSR- Vorsitzenden und Finanzreferent keine finanziell erheblichen Vorhaben beschließen. Der FSR kann mit der Mehrheit der gewählten Mitglieder Ausgaben beschließen, sofern der FSR-Sprecher oder der Finanzreferent mit der Mehrheit stimmen.
\end{enumerate}

\part{Schlussbestimmungen}\label{d.-schlussbestimmungen}

\section{Satzungsänderung}\label{satzungsuxe4nderung}

\begin{enumerate}[(1)]
	\item Diese Satzung behält ihre Gültigkeit bis sich die Fachschaft eine neue Satzung gibt.
	\item Die FSV oder die FSVV kann Änderungen dieser Satzung mittels Änderungssatzung beschließen.
	\item Dieser Beschluss muss jedes Mal von mindestens 2/3 der satzungsmäßigen FSV-Mitglieder bzw. von 2/3 der FSVV-Mitglieder gefasst werden. Die Regelungen zu außerordentlichen FSV- und FSVV- Sitzungen sind unanwendbar (§ 24 Abs. 5).
	\item §1, §6, §12 Abs. 2, §23, §33 und §34 können nicht aufgrund eines Beschlusses der FSV, sondern nur aufgrund des Beschlusses einer FSVV gemäß Abs. 3 geändert werden.
	\item Der Tagesordnungspunkt „Satzungsneufassung`` oder „Satzungsänderung`` muss bereits in der Einladung zur betreffenden FSVV-Sitzung angekündigt werden. Dem Einladungsschreiben ist der Wortlaut der beantragten Satzungsneufassung oder Änderungssatzung beizufügen.
	\item Die Satzung tritt mit ihrer Veröffentlichung in der AKUT in Kraft. Diese ist unverzüglich der Fachschaft auf einem geeigneten Kommunikationsweg bekanntzugeben.
\end{enumerate}

\end{document}
